\pdfbookmark[1]{Resumen}{Resumen}
\begingroup

\chapter*{Resumen}

El clustering siempre ha sido una potente herramienta en tareas de descubrimiento de conocimiento. Tradicionalmente en el marco del apredizaje no supervisado, recibió atención renovada cuando se le incorporaron nuevos tipos de información, dando lugar a un nuevo tipo de aprendizaje semisupervisado: el clustering con restricciones. Esta técnica es una generalicación del clustering clásico para considerar información adicional, dada en forma de restricciones. Estas pueden venir dadas en forma de restricciones a nivel de instancia tipo \textit{must-link} y \textit{cannot-link}, en las que este trabajo se centra. Proponemos la primera aplicación de Differential Evolution al clustering con restricciones, así como una nueva variante de la Búsqueda Local, tanto su esquema de optimización general como su aplicación al clustering con restricciones. Compararemos los resultados obtenidos por estas nuevas propuestas con los obtenidos por el estado del arte sobre veinte conjuntos de datos con niveles incrementales de infomación basada e restricciones, apoyando nuestras conclusiones con la ayuda de los test Bayesianos. \\


\noindent\textbf{Keywords:} Clustering con restricciones, restricciones a nivel de instancia, \textit{must-link}, \textit{cannot-link}, \textit{differential evolution}, búsqueda local.


\endgroup