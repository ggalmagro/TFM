\pdfbookmark[1]{Resumen}{Resumen}
\begingroup

\chapter*{Resumen}

El clustering siempre ha sido una potente herramienta en tareas de análisis de datos. Aunque tradicionalmente se ha ubicado dentro del marco del aprendizaje no supervisado, su potencial se vio ampliado tras la incorporación de nuevos tipos de información al mismo, dando lugar a un nuevo tipo de aprendizaje semisupervisado: el clustering con restricciones. Esta técnica es una generalización del clustering que considera información adicional, dada en forma de restricciones. Éstas pueden venir especificadas a nivel de instancia, centrándose este trabajo en las de tipo \textit{must-link} y \textit{cannot-link}. En este trabajo proponemos la primera aplicación de \textit{Differential Evolution} al clustering con restricciones, así como una nueva variante de la Búsqueda Local, incluyendo su esquema de optimización general y su aplicación al clustering con restricciones. Compararemos los resultados obtenidos por estas nuevas propuestas con los obtenidos por el estado del arte sobre veinte conjuntos de datos con niveles incrementales de información basada en restricciones. Apoyaremos nuestras conclusiones mediante el uso de tests Bayesianos. \\


\noindent\textbf{Palabras Clave:} Clustering con restricciones, restricciones a nivel de instancia, \textit{must-link}, \textit{cannot-link}, \textit{Differential Evolution}, Búsqueda Local.


\endgroup