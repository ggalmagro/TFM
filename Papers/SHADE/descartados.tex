%%%%%%%%%%%%%%%%% Test frecuentiastas %%%%%%%%%%%%%%%%%%%

\subsection{Applying the Wilcoxon's signed-rank test to our results}

We will apply the Wilcoxon's signed-rank test for each of the measures obtained, namely: ARI, Unsat and Time. We set as null hypothesis $H_0: $ the distribution of $A - B$ is symmetric about $\mu$. However, given that in the case of ARI it is a problem of maximisation, and in the case of Time and Unsat it is a problem of minimisation, we will consider two different alternative hypotheses. In the case of maximization we propose $H_{max}:$ the distribution of $A - B$ is shifted to the left with respect to $\mu$. On the other hand, in the case of minimization we propose $H_{min}:$ the distribution of $A - B$ is shifted to the right with respect to $\mu$. In this way we assume that SHADE+LS is better than BRKGA+LS and we try to prove it.

In this case we set $\mu = 0$, and work with a confidence level of 95\%, which means that the significance level is 0.05 ($\alpha = 0.05$). Table \ref{tab:freqTest} shows the statistical values obtained using Wilcoxon's signed-rank test for each comparison.

\begin{table}[!htp]
	\centering
	\setlength{\tabcolsep}{7pt}
	\renewcommand{\arraystretch}{1.4}
	\begin{tabular}{ccccc}
		\hline
		Measure & $R^{+}$ & $R^{-}$ & P-value & Alternative hypothesis  \\
		\hline
		ARI & 3088 & 1665 & 0.005254 & $H_{max}$ \\
		Time & 4008 & 1042 & 0.0000001723 & $H_{min}$ \\
		Unsat & 3082.5 & 745.5 & 0.0000003838 & $H_{min}$ \\
		\hline
	\end{tabular}
	\caption{Results obtained by the Wilcoxon's signed-rank test for the comparison SHADE vs BRKGA}
	\label{tab:freqTest}
\end{table}

For all three considered measures, $P-value < \alpha$, therefore in all cases we can reject $H_0$ and infer that there are significant differences between the two methods, giving the advantage to SHADE+LS with basis to the proposed alternative hypothesis. In conclusion, we can safely admit that [the] Wilcoxon's signed-rank test provides empirical evidence in favour of SHADE+LS.


-----------------------------------------------


Constrained clustering is an SSL learning method whose goal is to find a partition of the dataset that meets the proper characteristics of a clustering method result, in addition to satisfying a certain constraint set. It has been succesfully applied in many knowledge fields. It has been used to guide the movement of walking robots, as well as in other advanced robotics applications \cite{davidson2005clustering, semnani2016constrained}. In \cite{seret2014new} constrained clustering is presented as a useful tool in the context of applied marketing. Biology has also made use of constrained clustering, being able to detect simultaneous appearances of genes and proteins in biological data bases \cite{segal2003discovering}. In \cite{levy2008structural} a method of segmenting musical audio into separated sections using constrained clustering is described. Constrained clustering has also been useful to classify Internet traffic, which is considered to be one of the most fundamental functionalities in the network management field \cite{wang2014internet}. Electoral district design problems can also be approached using constrained clustering, as shown in \cite{brieden2017constrained}. Lastly, it is mandatory to name the widely known application to GPS data of constrained clustering, presented in \cite{wagstaff2001constrained}.