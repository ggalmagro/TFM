\documentclass[review]{elsarticle}

\usepackage{lineno,hyperref}
\usepackage{mathtools}
\usepackage{multirow}
\usepackage{adjustbox}
\usepackage{chngpage}
\usepackage[usenames,dvipsnames,svgnames,table]{xcolor}
\modulolinenumbers[5]

\journal{Applied Soft Computing}

%%%%%%%%%%%%%%%%%%%%%%%
%% Elsevier bibliography styles
%%%%%%%%%%%%%%%%%%%%%%%
%% To change the style, put a % in front of the second line of the current style and
%% remove the % from the second line of the style you would like to use.
%%%%%%%%%%%%%%%%%%%%%%%

%% Numbered
%\bibliographystyle{model1-num-names}

%% Numbered without titles
%\bibliographystyle{model1a-num-names}

%% Harvard
%\bibliographystyle{model2-names.bst}\biboptions{authoryear}

%% Vancouver numbered
%\usepackage{numcompress}\bibliographystyle{model3-num-names}

%% Vancouver name/year
%\usepackage{numcompress}\bibliographystyle{model4-names}\biboptions{authoryear}

%% APA style
%\bibliographystyle{model5-names}\biboptions{authoryear}

%% AMA style
%\usepackage{numcompress}\bibliographystyle{model6-num-names}

%% `Elsevier LaTeX' style
\bibliographystyle{elsarticle-num}
%%%%%%%%%%%%%%%%%%%%%%%
\newtheorem{definition}{Definition}
\begin{document}

\begin{frontmatter}

\title{Enhancing instance-level constrained clustering through differential evolution}

\author[mymainaddress]{Germ\'an Gonz\'alez Almagro\corref{mycorrespondingauthor}}
\cortext[mycorrespondingauthor]{Corresponding author}
\ead{germangalmagro@ugr.es}

\author[mymainaddress]{Salvador Garc\'ia}

\address[mymainaddress]{Department of Computer Science and Artificial Intelligence, University of Granada, 18071 Granada, Spain}

\begin{abstract}
This template helps you to create a properly formatted \LaTeX\ manuscript.
\end{abstract}

\begin{keyword}
\texttt{elsarticle.cls}\sep \LaTeX\sep Elsevier \sep template
\MSC[2010] 00-01\sep  99-00
\end{keyword}

\end{frontmatter}

\linenumbers

\section{Introduction}

\section{Background}

\subsection{Constrained Clustering}

Clustering is the task of grouping instances of a dataset into subsets. The criterion used to assign an instance to a given cluster is the similarity to the rest of elements in that cluster, and the dissimilarity to the rest of instances of the dataset. In this way, the clustering process obtains subsets of instances of the initial data set that present high intra-similarity and low inter-similarity.

In most clustering applications it is common to have some kind of information about the data set to be analyzed. In constrained clustering this information is given in the form of pairs of instances. A constraint states whether the instances which it refers to must, or must not, be assigned to the same cluster. Using this type of information it is possible to obtain a better result than by using completely unsupervised clustering algorithms.

Before formalizing the definition of constraints we must give a notation for the data set. We will note our dataset as $X$, which is a set of $N$ instances with $D$ features. We well refer to the instances of $X$ as $x$. Now, given the definition of dataset, we can distinguish between two types of constraints:

\begin{itemize}

	\item Must-link constraints $c_=(x_j,x_i)$: instances $x_i$ and $x_j$ from $X$ must be placed in the same cluster.

	\item Cannot-link constraints $c_{\neq}(x_i,x_j)$: instances $x_i$ and $x_j$ from $X$ cannot be assigned to the same cluster.

\end{itemize}

The goal of constrained clustering is to find a partition (or clustering) of $K$ clusters $C = {c_1 \cdots c_K}$ of the dataset $X$ that ideally satisfies all constraints in the constraint set. As in the original clustering problem, it must be fulfilled that the sum of instances in each cluster $c_i$ is equal to the number of instances in $X$, which we have defined as $N = |X| = \sum_{i = 1}^{K} |c_i|$.

Knowing how a constraint is defined, Must-link constraints---from now on ML---are an example of an equivalence relation; therefore, ML constraints are reflexive, transitive and symmetric. This way, given constraints $c_=(x_a,x_b)$ and $c_=(x_b,x_c)$ then $c_=(x_a,x_c)$ is verified. In addition, if $x_a \in c_i$ and $x_b \in c_j$ are related by $c_=(x_a,x_b)$, then $c_=(x_c,x_d)$ is verified for any $x_c \in c_i$ and $x_d \in c_j$ \cite{xu2013improving}\cite{davidson2007survey}.

On the other hand, Cannot-link constraints---from now on CL---do not constitute an equivalence relation***, but they also can be entailed***. However, analogously, given $x_a \in c_i$ and $x_b \in c_j$, and the constraint $c_{\neq}(x_a,x_b)$, then it is also true that $c_{\neq}(x_c,x_d)$ for any $x_c \in c_i$ and $x_d \in c_j$ \cite{davidson2007survey}.

\subsection{The Feasibility Problem}

Given that constrained clustering adds a new element to the clustering problem, we must consider how this element affects the problem complexity. The feasibility problem for constrained clustering was defined by \cite{davidson2005clustering} as in Definition \ref{def1}.

\begin{definition}
	
	\textbf{Feasibility Problem}: given a dataset $X$, a set of constraints $R$, and the bounds on the number of clusters $K_l \leq k \leq K_u$. Does there exists a partition $C$ of $X$ with $k$ clusters such that all constraints in $R$ are satisfied? \cite{davidson2005clustering}
	\label{def1}
	
\end{definition}

In \cite{davidson2005clustering} it is proven that, when $K_l = 1$ and $K_u \ge 3$, the feasibility problem for constrained clustering is $\mathbf{NP}$-complete, by reducing it from the Graph K-Colorability problem. (It is also proven that it is not harder than it, so both have the same complexity.) Table \ref{tab:feasibility} shows feasibility problem complexity for different types of constraints.

\begin{table}[!h]
	\centering
	%\setlength{\arrayrulewidth}{1mm}
	\setlength{\tabcolsep}{7pt}
	\renewcommand{\arraystretch}{0.9}
	%\resizebox{\textwidth}{!}{
		\begin{tabular}{c c}
			\hline
			Constraints & Complexity \\
			\hline
			Must-Link & $\mathbf{P}$\\
			Cannot-Link & $\mathbf{NP}$-complete\\
			ML and CL & $\mathbf{NP}$-complete\\
			\hline
			
		\end{tabular}%}
	\caption{Feasibility problem complexity \cite{davidson2005clustering}}
	\label{tab:feasibility}
\end{table}

These complexity results show that feasibility problem with CL constraints is intractable and hence constrained clustering is intractable too. For more details on constrained clustering problem complexity see \cite{davidson2005clustering}.

Intractable problems are hard to solve with deterministic and exact methods. That is the reason why applying heuristics-guided and population-based algorithms is a good approach to find quality solutions to the constrained clustering problem.


\section{Brief Review of the BRKGA Algorithm}

The biased random-key genetic algorithm---from now on BRKGA---was first proposed by \cite{gonccalves2011biased} as a generalization of the random-key genetic algorithm \cite{bean1994genetic}.

In BRKGA, each solution---also called individual---is represented as a vector of values within the interval $[0,1]$. However, this vector is not a solution to the problem but a representation of it. This allows us to abstract the details of the problem in order to apply a genetic algorithm that operates with real coding. Therefore, a deterministic decoder is necessary to obtain the actual solution to the problem from the vector that encodes it.

BRKGA also requires a fitness function to evaluate each individual. This function takes the decoded solution as input and provides its fitness value as output. It is designed specifically for each problem.

The BRKGA optimization process is completely independent of the problem to solve. First, a population $P$ of $p_{size}$ vectors of random-keys is initialized, and this will be the initial generation. Each random-key vector $p_i$ has $N$ random-keys in it. The population is sorted by the fitness value $f_i$ of each $p_i$ to select the first $p_e$ individuals, this is, the elite of the population. The elite will be preserved in the next generation without modification. To introduce diversity, a number $p_m$ of new random-keys vectors are also included in the next generation. The remaining individuals of the new generation ($p_{size} - p_e - p_m$) are obtained through crossovers between elite and non-elite parents \cite{de2017comparison}.

\subsection{Adaptation of BRKGA for Constrained Clustering} \label{sec:AdaptationofBRKGA}

An adaptation of BRKGA for constrained clustering was proposed in \cite{de2017comparison}.

Their method randomly initializes a population of random-key vectors, each one with $N$ random-keys. The decoder divides the interval $[0,1]$ in $K$ intervals, so there exists a correspondence between each random-key and the integer corresponding to the interval which it lies in. Table \ref{tab:decodingrk} shows an example of random-key decoding for $K = 3$. Note that extreme values 0 and 1 can also appear in a random-key vector.

\begin{table}[!h]
	\centering
	%\setlength{\arrayrulewidth}{1mm}
	\setlength{\tabcolsep}{7pt}
	\renewcommand{\arraystretch}{0.9}
	\resizebox{\textwidth}{!}{
	\begin{tabular}{|c|c|c|c|c|c|c|c|c|c|c|}
		\hline
		Index & 1 & 2 & 3 & 4 & 5 & 6 & 7 & 8 & 9 & 10 \\
		\hline
		Random-key & 0.12 & 0.37 & 0.66 & 0.56 & 0.00 & 0.97 & 0.23 & 0.25 & 0.15 & 1.00 \\
		\hline
		Clusters & 1 & 2 & 2 & 2 & 1 & 3 & 1 & 3 & 1 & 3 \\
		\hline

	\end{tabular}}
	\caption{Random-key decodification example \cite{de2017comparison}}
	\label{tab:decodingrk}
\end{table}

Once the decoded solution has been obtained, the fitness value of the solution is computed as in Equation \eqref{eq1}.

\begin{equation}
f_i = z_i + \overbrace{(\mu * N * {infeasibility}_i)}^{penalty}
\label{eq1}
\end{equation}

Where $\mu$ is a high value, $infeasibility_i$ is the number of non-satisfied constraints and $z_i$ is the within-cluster-sum-of-squares which can be computed as in Equation \eqref{eq2}.

\begin{equation}
z_i = \sum_{c_j \in C_i} \left[ \frac{\sum_{x_a, x_b \in c_j} d_D^2(x_a,x_b)}{|c_j|}\right]
\label{eq2}
\end{equation}

Where $C_i$ is the partition of dataset $X$ defined by solution $p_i$, and $(x_a, x_b)$ are instances in the cluster $c_j$ such that $a \neq b$ and the distance between each pair of instances in $c_i$ is included in the sum only once.

In \cite{de2017comparison} authors also added a new element to BRKGA, a local search procedure. This local optimization method is applied to each decoded individual in each generation, but its results are not transferred to the individual, so that diversity is maintained. Thus, the individuals produced by the local search are only used to update the best solution found so far if needed. For more details on BRKGA see \cite{de2017comparison}.

\section{The SHADE Optimization Method}

Success history based adaptive differential evolution---from now on SHADE---is an optimization method based on differential evolution. Proposed by \cite{tanabe2013success}, it was an improvement for the JADE model, created by \cite{zhang2009jade}.

Unlike JADE, which only considers the parameters used in the previous generation, SHADE uses a historical record of successful parameters as a mechanism for adapting parameters involved in the process of creating new generations.

\subsection{SHADE Operators}

As in differential evolution, SHADE employs mutant generation, crossover and replacement operators to explore the space of solutions and bring in new generations of individuals.

The mutation strategy used by SHADE is known as current-to-pbest/1, and the expression that generates new individuals is described in Equation \eqref{eq3}.

\begin{equation}
m_{i,G} = p_{i,G} + F_i * (p_{pbest, G} - p_{i,G}) + F_i * (p_{r1, G} - p_{r2,G})
\label{eq3}
\end{equation}

Where $m_{i,G}$ is the mutant vector, which is generated with an individual $p_{i,G}$ from the population [serving as a starting point]. Indices $r1$ and $r2$ are random values in the range $[0,N]$, different from $i$ and from each other. The individual $p_{pbest, G}$ is randomly selected from among the best $N \times pbest\;|\;pbest\in [0,1]$ individuals in the population. This way, the parameter $p$ controls the greediness of the mutation strategy. The parameter $F_i$ defines the magnitude of the operator.

After generating the mutant vector $m_{i,G}$, it is combined with the parent $p_{i,G}$ vector by means of a crossover operator; the result is the trial vector. SHADE uses the Binomial crossover operator, which is defined by the Equation \eqref{eq4}.

\begin{equation}
t_{j,i,G} = \left\{ \begin{array}{lc}
m_{j,i,G} &   if \;\; rand[0,1) \le CR_i \;\; or \;\;j = j_{rand} \\
p_{j,i,G} &  otherwise
\end{array}
\right.
\label{eq4}
\end{equation}

Where $rand[0,1)$ is a random number selected from a normal distribution in the range $[0,1)$, $j_{rand}$ is a random integer selected from the range $[1,D]$, and $CR \in [0,1]$ is the crossover ratio.

The SHADE replacement operator determines the survivors individuals for the next generation. It compares each parent $x_{i,G}$ with the trial vector $_{j,i,G}$, generated based on it. The best one of these two individuals will survive for the next generation. Equation \eqref{eq5} shows this concept.

\begin{equation}
p_{i,G + 1} = \left\{ \begin{array}{lc}
t_{i,G} &   if \;\; f(t_{i,G}) \le f(p_{i,G}) \\
p_{i,G} &  otherwise
\end{array}
\right.
\label{eq5}
\end{equation}

To maintain diversity, SHADE can make use of an external archive $A$, in which parents $p_{i,G}$ who were replaced by their associated trial vector $t_{i,G}$ are saved. To make use of the archive, we will consider that the individual $p_{r2,G}$ from Equation \eqref{eq3} is selected from $P \cup A$. The archive size is the same as the population size. When the size of $A$ exceeds the size of $P$, random individuals are selected for removal to make room for new ones.

The parameters $p$, $F_i$ and $CR_i$ are difficult to adjust and their selection is not trivial, as they largely determine the success of the optimization process. These are the parameters that SHADE adaptively optimizes using the mentioned history record.

\subsection{SHADE Parameter Adaptation Method}

The SHADE method stores in memory a structure with $H$ entries for parameters $F_i$ and $CR_i$, as shown in Table \ref{tab:SHADEmemory}. This structure is initialized with the value $0.5$ for all entries.

\begin{table}[!h]
	\centering
	%\setlength{\arrayrulewidth}{1mm}
	\setlength{\tabcolsep}{13pt}
	%\renewcommand{\arraystretch}{0.9}
	\resizebox{\textwidth}{!}{
	\begin{tabular}{|c|c|c|c|c|c|}
		\hline
		 Index & 1 & 2 & $\cdots$ & $H - 1$ & $H$ \\
		 \hline
		 \hline
		 $M_{CR}$ & $M_{CR,1}$ & $M_{CR,1}$ & $\cdots$ & $M_{CR,H-1}$ & $M_{CR,H}$ \\
		 \hline
		 $M_{F}$ & $M_{F,1}$ & $M_{F,1}$ & $\cdots$ & $M_{F,H-1}$ & $M_{F,H}$ \\
		\hline

	\end{tabular}}
	\caption{Historical memory $M_{CR}$, $M_{F}$ used by SHADE \cite{tanabe2013success}}
	\label{tab:SHADEmemory}
\end{table}

In each generation, parameters $CR_i$ and $F_i$ are calculated on the basis of the existing history by applying Equations \eqref{eq6} and \eqref{eq7}:

\begin{equation}
CR_i = randn_i(M_{CR,r_i}, 0.1)
\label{eq6}
\end{equation}

\begin{equation}
F_i = randc_i(M_{F,r_i}, 0.1)
\label{eq7}
\end{equation}

where $randn(\mu, \sigma^2)$ and $randc(\mu, \sigma^2)$ are random values from normal and Cauchy distributions respectively, with mean $\mu$ and variance $\sigma^2$. When a value for $CR_i$ outside of the range $[0,1]$ is generated, it is replaced by the corresponding limit value. If $F_i > 1$, then it is truncated to $1$; conversely, if $F_i < 0$, then Equation \eqref{eq7} is applied as many times as needed to obtain a legal value.

SHADE keeps two auxiliar sets $S_{CR}$ and $S_F$ to store the $CR_i$ and $F_i$ values that were successfully used to generate a trial vector that replaced the parent. At the end of each generation, the content of the memory $M$ is updated following Equations \eqref{eq8} and \eqref{eq9}.

\begin{equation}
M_{CR,h,G+1} = \left\{ \begin{array}{lc}
mean_{WA} (S_{CR}) &   if \;\; S_{CR} \neq \emptyset \\
M_{CR,h,G+1} &  otherwise
\end{array}
\right.
\label{eq8}
\end{equation}

\begin{equation}
M_{F,h,G+1} = \left\{ \begin{array}{lc}
mean_{WL} (S_{F}) &   if \;\; S_{F} \neq \emptyset \\
M_{F,h,G+1} &  otherwise
\end{array}
\right.
\label{eq9}
\end{equation}

The $h \;\; (0 \le h \le H)$ index specifies the position of the memory $M$ to be updated. This index is initialized to $0$ at the beginning of the optimization process and increased by one after each generation. When $h \ge H$ then $h$ is reset to 1. It is worth noting that, when there is a generation with no trial vectors successfully replacing their parents, no update of $M$ is done.

In Equation \ref{eq8}, the term $mean_{WA} (S_{CR})$ is the weighted mean, which is computed following Equations \eqref{eq10} and \eqref{eq11}, proposed by \cite{peng2009multi} to prevent $CR$ from converging to small values.

\begin{equation}
mean_{WA} (S_{CR}) = \sum_{i = 1}^{|S_{RC}|} \omega_i * S_{RC,i}
\label{eq10}
\end{equation}

\begin{equation}
\omega_i = \frac{\Delta f_i}{\sum_{i = 1}^{|S_{RC}|} \Delta f_i}
\label{eq11}
\end{equation}

where $\Delta f_i = |f(t_i,G) - f(p_i, G)|$, which is the amount of improvement obtained from the trial vector with respect to the parent.

In Equation \ref{eq9} the term $mean_{WL} (S_{F})$ refers to the weighted Lehmer mean, which is computed as in Equation \eqref{eq12} (proposed by \cite{tanabe2013success}).

\begin{equation}
mean_{WL} (S_{F}) = \frac{\sum_{i = 1}^{|S_{F}|} \omega_i * S^2_{F,i}}{\sum_{i = 1}^{|S_{F}|} \omega_i * S_{F,i}}
\label{eq12}
\end{equation}

Unlike parameters $S_{CR}$ and $S_F$, the parameter $pbest$ from Equation \eqref{eq3}, used to set the qreediness of the mutation strategy, is not included in the optimization process. However, it is not static: it is calculated for each individual $p_i$ of the population following Equation \eqref{eq13}.

\begin{equation}
pbest_i = randn[2/N, 0.2]
\label{eq13}
\end{equation}

such that there is always at least two individuals to choose between.

\subsection{Adaptation of SHADE for constrained clustering}

For the adaptation of SHADE to the problem of constrained clustering, we will take some of the elements that allowed us to adapt it to BRKGA. To begin with, we will use random-keys to create the individuals of the population; this way, each individual is defined as a vector of random-keys. Additionally, we will also reuse the same initialization method and the same decoder used in BRKGA (see Section \ref{sec:AdaptationofBRKGA}).

However, we propose a new fitness function to evaluate the quality of the individuals of the population, Equation \eqref{eq14}, whose element $z_i$ is obtained as in Equation \eqref{eq2}.

\begin{equation}
f_i = z_i * (infeasibility + 1)
\label{eq14}
\end{equation}

Note that in Equation \eqref{eq14} no parameter is involved that cannot be calculated from the dataset or constraints, whereas in \eqref{eq1} the $\mu$ parameter must be calculated and optimized for each individual problem.

We found that with the fitness function defined in \eqref{eq1} there is no competition between the penalty term and the within-cluster-sum-of-squares term, because the penalty is always significantly larger than it or zero. This can bias the exploration of the solution space, restricting it in practice to those that satisfy all the constraints, even if the within-cluster-sum-of-squares is still improvable by moving two instances involved in a constraint to different clusters without violating that constraint.

With Equation \eqref{eq14} we try to find a trade-off between the within-cluster-sum-of-squares and the penalty term, allowing solutions that violate a certain number of constraints to compete with those who violate a smaller number of them but score a better within-cluster-sum-of-squares.

\textcolor{red}{Poner una pintura que ejemplifique lo anterior.}\\
\textcolor{red}{Explicar la busqueda local. Con pseudocódigo?}

\clearpage

\section{Experimental Setup}

\subsection{Dataset}

\begin{table}[!h]
	\centering
	%\setlength{\arrayrulewidth}{1mm}
	\setlength{\tabcolsep}{5pt}
	\renewcommand{\arraystretch}{0.8}
	%\resizebox{\textwidth}{!}{
	\small
	\begin{tabular}{l c c c}
		\hline
		Name & No. Instances & No. Classes & Features \\
		\hline
		Appendicitis & 106 & 2 &  \\
		Balance & 625 & 3 &  \\
		Boston & 506 & 229 &  \\
		Breast Cancer & 569 & 2 &  \\
		Bupa & 345 & 2 &  \\
		Circles & 300 & 2 &  \\
		Contraceptive & 1473 & 3 &  \\
		Diabetes & 442 & 214 &  \\
		Ecoli & 336 & 8 &  \\
		Glass & 214 & 6 &  \\
		Haberman & 306 & 2 &  \\
		Hayesroth & 160 & 3 &  \\
		Heart & 270 & 2 &  \\
		Ionosphere & 351 & 2 &  \\
		Iris & 150 & 3 &  \\
		Led7Digit & 500 & 10 &  \\
		Monk2 & 432 & 2 &  \\
		Moons & 300 & 2 &  \\
		Movement Libras & 360 & 15 &  \\
		Newthyroid & 215 & 3 &  \\
		Pima & 768 & 2 &  \\
		Rand & 150 & 3 &  \\
		Saheart & 462 & 2 &  \\
		Sonar & 208 & 2 &  \\
		Soybean & 47 & 4 &  \\
		Spectfheart & 267 & 2 &  \\
		Spiral & 300 & 2 &  \\
		Tae & 151 & 3 &  \\
		Vehicle & 846 & 4 &  \\
		Vowel & 990 & 11 &  \\
		Wdbc & 569 & 2 &  \\
		Wine & 178 & 3 &  \\
		Zoo & 101 & 7 &  \\
		\hline

	\end{tabular}%}
	\caption{Summary of datasets used for the experiments \textcolor{red}{Citar keel}}
	\label{tab:datasets}
\end{table}

\subsection{Constraints Generation}

Since we have the true labels associated with each data set, we will use the method proposed by Wagstaff and Cardie to generate artificial constraint sets \cite{wagstaff2001constrained}. This method consists in randomly selecting two instances of a data set, then comparing its labels, and finally setting an ML or CL constraint depending on whether the labels are the same or different.

We will generate four different sets of constraints for each data set. The number of constraints is given by the number of edges of the complete constraints graph that we can build with a percentage of labeled data. Each constraint set is associated with a different percentage of labeled data, namely: 5\%, 10\%, 15\% and 20\%. It is worth noting that constraints are not the complete constraint graph made up of the tagged instances, this graph only sets the number of constraints to be created, this way there is a lower probability of biasing the set of constraints so that there are classes with poor representation in it.

\begin{table}[!h]
	\centering
	\setlength{\tabcolsep}{7pt}
	\renewcommand{\arraystretch}{0.8}
	%\begin{adjustwidth}{-1in}{-1in}
	\resizebox{\textwidth}{!}{
	\begin{tabular}{lcc c cc c cc c cc}
		\hline
		\multirow{2}{*}{Dataset} &
		\multicolumn{2}{c}{5\%} &&  \multicolumn{2}{c}{10\%} && \multicolumn{2}{c}{15\%} && \multicolumn{2}{c}{20\%} \\
		\cline{2-3} \cline{5-6} \cline{8-9} \cline{11-12}
		& ML & CL && ML & CL && ML & CL && ML & CL \\
		\hline
		Appendicitis & 13 & 2 && 39 & 16 && 71 & 49 && 164 & 67 \\
		Balance & 198 & 298 && 838 & 1115 && 1865 & 2506 && 3336 & 4414 \\
		Boston & 0 & 325 && 3 & 1272 && 13 & 2837 && 26 & 5125 \\
		Breast Cancer & 216 & 190 && 876 & 720 && 1965 & 1690 && 3487 & 2954 \\
		Bupa & 79 & 74 && 323 & 272 && 699 & 627 && 1201 & 1145 \\
		Circles & 50 & 55 && 208 & 227 && 502 & 488 && 853 & 917 \\
		Contraceptive & 994 & 1707 && 3765 & 7113 && 8501 & 15809 && 15399 & 27966 \\
		Diabetes & 2 & 251 && 5 & 985 && 7 & 2204 && 23 & 3893 \\
		Ecoli & 30 & 106 && 163 & 398 && 357 & 918 && 609 & 1669 \\
		Glass & 11 & 44 && 52 & 179 && 139 & 389 && 259 & 644 \\
		Haberman & 76 & 44 && 304 & 161 && 634 & 401 && 1135 & 756 \\
		Hayesroth & 12 & 16 && 39 & 81 && 102 & 174 && 177 & 319 \\
		Heart & 41 & 50 && 178 & 173 && 396 & 424 && 744 & 687 \\
		Ionosphere & 92 & 61 && 330 & 300 && 732 & 646 && 1299 & 1186 \\
		Iris & 9 & 19 && 26 & 79 && 82 & 171 && 136 & 299 \\
		Led7Digit & 25 & 275 && 126 & 1099 && 267 & 2508 && 460 & 4490 \\
		Monk2 & 101 & 130 && 473 & 473 && 979 & 1101 && 1917 & 1824 \\
		Moons & 55 & 50 && 200 & 235 && 494 & 496 && 900 & 870 \\
		Movement Libras & 6 & 147 && 27 & 603 && 112 & 1319 && 158 & 2398 \\
		Newthyroid & 25 & 30 && 108 & 123 && 270 & 258 && 449 & 454 \\
		Pima & 412 & 329 && 1604 & 1322 && 3595 & 3075 && 6452 & 5329 \\
		Rand & 8 & 20 && 25 & 80 && 76 & 177 && 151 & 284 \\
		Saheart & 152 & 124 && 595 & 486 && 1292 & 1123 && 2330 & 1948 \\
		Sonar & 29 & 26 && 100 & 110 && 245 & 251 && 436 & 425 \\
		Soybean & 0 & 3 && 4 & 6 && 6 & 22 && 12 & 33 \\
		Spectfheart & 56 & 35 && 233 & 118 && 543 & 277 && 965 & 466 \\
		Spiral & 52 & 53 && 224 & 211 && 487 & 503 && 918 & 852 \\
		Tae & 8 & 20 && 40 & 80 && 82 & 171 && 151 & 314 \\
		Vehicle & 221 & 682 && 874 & 2696 && 1955 & 6046 && 3589 & 10776 \\
		Vowel & 107 & 1118 && 445 & 4406 && 1026 & 10000 && 1705 & 17798 \\
		Wdbc & 209 & 197 && 840 & 756 && 1925 & 1730 && 3472 & 2969 \\
		Wine & 14 & 22 && 49 & 104 && 121 & 230 && 217 & 413 \\
		Zoo & 7 & 8 && 21 & 34 && 29 & 91 && 41 & 169 \\
		\hline

	\end{tabular}}
	%\end{adjustwidth}

	\caption{Number of constraints used in experiments}
	\label{tab:constraints}
\end{table}

\subsection{Evaluation Method}

We use the evaluation method proposed by Wagstaff et al. (2001) \cite{wagstaff2001constrained}, the authors of the Constrained K-Means algorithm, proposed in their work to evaluate the results.

Since we have the true labels associated to each of the datasets, we can use them in post-processing to evaluate the results provided by each method. We will use Rand Index \cite{rand1971objective} to measure the accuracy of the predictions resulting from each method we test. Rand Index computes de degree of agreement between two partitions $C_1$ and $C_2$ of a given dataset $X$. $C_1$ and $C_2$ are viewed as collections of $N(N - 1)/2$ pairwise decisions.

For each pair of $x_i$ and $x_j$ instances in $X$, $C_i$ assigns them to the same cluster or to different clusters. We take $a$ as the number of pairings where $x_i$ is in the same cluster as $x_j$ in $C_1$ and $C_2$ , and $b$ as the opposite event. Then, the degree of similarity between $C_1$ and $C_2$ is calculated as in Equation \eqref{eq15}.

\begin{equation}
Rand(C_1, C_2) = \frac{a + b}{N(N - 1)/2}
\label{eq15}
\end{equation}

We will use this measure co compute the accuracy for all of our experiments.

\subsection{Calibration}

\section{Experimental Results}

\begin{table}[!h]
	\centering
	\setlength{\tabcolsep}{7pt}
	\renewcommand{\arraystretch}{0.9}
	%\begin{adjustwidth}{-1in}{-1in}
	\resizebox{\textwidth}{!}{
		\begin{tabular}{l ccc c ccc}
			\hline
			\multirow{2}{*}{Dataset} &
			\multicolumn{3}{c}{BRKGA} &&  \multicolumn{3}{c}{SHADE} \\
			\cline{2-4} \cline{6-8}
			& RandIndex & Unsat(\%) & Time(s) && RandIndex & Unsat(\%) & Time(s) \\
			\hline
			&  &  &  &&  &  &  \\
			&  &  &  &&  &  &  \\
			&  &  &  &&  &  &  \\
			\hline
			Means &  &  &  &&  &  &  \\
			\hline

		\end{tabular}}
		%\end{adjustwidth}

	\caption{Experimental results obtained with X\% of labeled data}
	\label{undefined}
\end{table}

\section{Conclusions and Future Work}

\section{Acknowledgements}

\clearpage

\section*{References}

\bibliography{mybibfile}

\end{document}
